\section{Исследование влияния распределения массы робота по колёсам}

{\bfseries Гипотеза:} отличие полученных характеристик для второго двигателя от первого и третьего вызвано неравномерным распределением массы робота. 
Конструкции робота предполагает расположение двух аккумуляторных батарей в задней части робота между первым и вторым и вторым и третьим колёсами, что создаёт дисбаланс распределения массы по колёсам.
Получается, что на второе колесо приходится больше массы, чем на первое и третье.

Для того, чтобы сделать распределение массы по колёсам более равномерным, добавим такую же аккумуляторную батарею в переднюю часть робота между первым и третьим колесом.

\begin{figure}[H]
    \centering
    \begin{subfigure}{0.49\textwidth}
        \centering
        \includegraphics[width=\textwidth]{wheel_load_distribution/polar_current/half_second_averaging/gray motor 1.pdf}
        \caption{двигатель 1}
    \end{subfigure}
    \hspace{0.005\textwidth}
    \begin{subfigure}{0.49\textwidth}
        \centering
        \includegraphics[width=\textwidth]{wheel_load_distribution/polar_current/half_second_averaging/gray motor 3.pdf}
        \caption{двигатель 3}
    \end{subfigure} \\
    \vspace{4pt}
    \centering
    \begin{subfigure}{0.49\textwidth}
        \centering
        \includegraphics[width=\textwidth]{wheel_load_distribution/polar_current/half_second_averaging/gray motor 2.pdf}
        \caption{двигатель 2}
    \end{subfigure}
    \caption{Зависимость тока двигателя от направления движения робота на серой поверхности для неравномерного и равномерного распределение массы}
\end{figure}

\begin{figure}[H]
    \centering
    \begin{subfigure}{0.49\textwidth}
        \centering
        \includegraphics[width=\textwidth]{wheel_load_distribution/polar_current/full_averaging/gray motor 1.pdf}
        \caption{двигатель 1}
    \end{subfigure}
    \hspace{0.005\textwidth}
    \begin{subfigure}{0.49\textwidth}
        \centering
        \includegraphics[width=\textwidth]{wheel_load_distribution/polar_current/full_averaging/gray motor 3.pdf}
        \caption{двигатель 3}
    \end{subfigure} \\
    \vspace{4pt}
    \centering
    \begin{subfigure}{0.49\textwidth}
        \centering
        \includegraphics[width=\textwidth]{wheel_load_distribution/polar_current/full_averaging/gray motor 2.pdf}
        \caption{двигатель 2}
    \end{subfigure}
    \caption{Зависимость среднего значения тока от направления движения робота}
\end{figure}

\begin{figure}[H]
    \centering
    \begin{subfigure}{0.49\textwidth}
        \centering
        \includegraphics[width=\textwidth]{wheel_load_distribution/robot_delta_speed_vs_axis_current/gray/x_axis.pdf}
        \caption{вдоль оси X}
    \end{subfigure}
    \hspace{0.005\textwidth}
    \begin{subfigure}{0.49\textwidth}
        \centering
        \includegraphics[width=\textwidth]{wheel_load_distribution/robot_delta_speed_vs_axis_current/gray/y_axis.pdf}
        \caption{вдоль оси Y}
    \end{subfigure} \\
    \vspace{4pt}
    \centering
    \begin{subfigure}{0.49\textwidth}
        \centering
        \includegraphics[width=\textwidth]{wheel_load_distribution/robot_delta_speed_vs_axis_current/gray/rotation.pdf}
        \caption{вращательное движение}
    \end{subfigure}
    \caption{Сравнение по осям локальной системы координат робота}
\end{figure}

\begin{figure}[H]
    \centering
    \begin{subfigure}{0.49\textwidth}
        \centering
        \includegraphics[width=\textwidth]{wheel_load_distribution/wheel_slippage_vs_motor_current/gray motor 1.pdf}
        \caption{двигатель 1}
    \end{subfigure}
    \hspace{0.005\textwidth}
    \begin{subfigure}{0.49\textwidth}
        \centering
        \includegraphics[width=\textwidth]{wheel_load_distribution/wheel_slippage_vs_motor_current/gray motor 3.pdf}
        \caption{двигатель 3}
    \end{subfigure} \\
    \vspace{4pt}
    \centering
    \begin{subfigure}{0.49\textwidth}
        \centering
        \includegraphics[width=\textwidth]{wheel_load_distribution/wheel_slippage_vs_motor_current/gray motor 2.pdf}
        \caption{двигатель 2}
    \end{subfigure}
    \caption{Сравнение по двигателям}
\end{figure}

\subsection{Измерение силы нормальной реакции опоры на колёсах Robotino 2}

Измерения проводились с помощью кухонных весов с диапазоном измерения \qtyrange{2}{5000}{г}.
Точность измерения весов \qty{1}{г}.

Для измерения силы нормальной реакции опоры или силы давления колеса на опору под одно из колёс робота подкладывались весы, под оставшиеся колёса -- опоры примерно равной с весами высотой.
Это необходимо, чтобы избежать перекоса робота и тем самым повысить точность полученных измерений.

Для проверки разброса значений измерителя был \num{5} раз последовательно взвешен балласт (один из аккумуляторов Robotino).
Балласт в данном случае -- это масса, которая устанавливается на робота в целях равномерно загрузить колёса робота.
Получились следующие показания: \qtylist{1387;1387;1387;1387;1386}{г}.

Были произведены два варианта измерений: без балласта и с балластом.
Показания измерений и оценки погрешности приведены в таблицах \ref{tab:nonuniform_load} и \ref{tab:uniform_load}.

\begin{table}[H]
    \centering
    \caption{Распределение массы робота по колёсам без балласта}
    \label{tab:nonuniform_load}
    \sisetup{
        round-mode = places,
        round-precision = 1
    }%
    \begin{tabular}{S[round-mode = none]SSS}
         & \multicolumn{3}{c}{Распределение массы по колёсам, \unit{г}} \\ \cmidrule(lr){2-4}
        {№ измерения} & {1-ое} & {2-ое} & {3-е} \\
        \midrule
        1 & 2812 & > 5200 & 3268 \\
        2 & 2593 & > 5200 & 3210 \\
        3 & 3433 & 4905 & 2904 \\
        4 & 3616 & 4922 & 2981 \\
        5 & 3045 & > 5200 & 3105 \\
        \midrule
        {среднее} & 3099.8 & 5085.4 & 3093.6 \\
        {среднеквадратическое отклонение} & 424.31 & 157.04 & 152.25 \\
        {доверительные границы (P = 0.95)} & 189.758 & 70.229 & 68.088 \\
    \end{tabular}
\end{table}

\begin{table}[H]
    \centering
    \caption{Распределение массы робота по колёсам с балластом}
    \label{tab:uniform_load}
    \sisetup{
        round-mode = places,
        round-precision = 1
    }%
    \begin{tabular}{S[round-mode = none]SSS}
         & \multicolumn{3}{c}{Распределение массы по колёсам, \unit{г}} \\ \cmidrule(lr){2-4}
        {№ измерения} & {1-ое} & {2-ое} & {3-е} \\
        \midrule
        1 & 3861 & 4663 & 4069 \\
        2 & 3353 & 4630 & 4238 \\
        3 & 3981 & > 5200 & 3517 \\
        4 & 3998 & > 5200 & 3762 \\
        5 & 4235 & > 5200 & 3527 \\
        \midrule
        {среднее} & 3885.6 & 4978.6 & 3822.6 \\
        {среднеквадратическое отклонение} & 327.140 & 303.388 & 323.153 \\
        {доверительные границы (P = 0.95)} & 294.804 & 273.401 & 291.212 \\
    \end{tabular}
\end{table}

Получены следующие оценки распределения массы робота/силы давления по колёсам:
\begin{itemize}
    \item без балласта
    \begin{itemize}
        \item первое колесо: \qty{3099.8 \pm 382.4}{г} или \qty{30.409 \pm 3.751}{Н};
        \item второе колесо: \qty{> 5085.4 \pm 141.5}{г} или \qty{> 49.889 \pm 1.388}{Н};
        \item третье колесо: \qty{3093.6 \pm 137.2}{г} или \qty{30.348 \pm 1.345}{Н};
    \end{itemize}
    \item с балластом
    \begin{itemize}
        \item первое колесо: \qty{3885.6 \pm 294.804}{г} или \qty{38.117 \pm 2.892}{Н};
        \item второе колесо: \qty{> 4978.6 \pm 273.401}{г} или \qty{> 48.840 \pm 2.682}{Н};
        \item третье колесо: \qty{3822.6 \pm 291.212}{г} или \qty{37.499 \pm 2.856}{Н};
    \end{itemize}
\end{itemize}